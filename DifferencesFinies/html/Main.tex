

% This document was generated by the publish-function
% from GNU Octave 5.1.0



\documentclass[10pt]{article}
\usepackage{listings}
\usepackage{mathtools}
\usepackage{amssymb}
\usepackage{graphicx}
\usepackage{hyperref}
\usepackage{xcolor}
\usepackage{titlesec}
\usepackage[utf8]{inputenc}
\usepackage[T1]{fontenc}
\usepackage{lmodern}


\lstset{
language=Octave,
numbers=none,
frame=single,
tabsize=2,
showstringspaces=false,
breaklines=true}


\titleformat*{\section}{\Huge\bfseries}
\titleformat*{\subsection}{\large\bfseries}
\renewcommand{\contentsname}{\Large\bfseries Contents}
\setlength{\parindent}{0pt}

\begin{document}

{\Huge\section*{Main}}

\tableofcontents
\vspace*{4em}

\begin{lstlisting}
close all
clear all
\end{lstlisting}


\phantomsection
\addcontentsline{toc}{section}{Declaration des variables}
\subsection*{Declaration des variables}

\begin{lstlisting}
global z k B1 B2 D n mu D2 dz;
\end{lstlisting}


\phantomsection
\addcontentsline{toc}{section}{Definition des parametres}
\subsection*{Definition des parametres}

\begin{lstlisting}
D = 0.01;
mu =  0.08 ;
k = 3;
i=0;
tmax= 200;
pas=0.2;
z0 = 0;
zL = 50;
n = 201;
B1 = 0.008;
B2 = 2.54*B1;
\end{lstlisting}


\phantomsection
\addcontentsline{toc}{section}{Creation de la grille spatio temporelle}
\subsection*{Creation de la grille spatio temporelle}

\begin{lstlisting}
dz = (zL - z0)/(n - 1);
z = z0:dz:zL;
z = z';
t=0:pas:tmax;
t = t';
\end{lstlisting}


\phantomsection
\addcontentsline{toc}{section}{Approximation de la derivee seconde}
\subsection*{Approximation de la derivee seconde}

\begin{lstlisting}
D2 = three_point_centered_D2(z);
\end{lstlisting}


\phantomsection
\addcontentsline{toc}{section}{Conditions initiales (vecteur initial)}
\subsection*{Conditions initiales (vecteur initial)}

\begin{lstlisting}
v0 = zeros (length(z),1);
w0 = zeros (length(z),1);
u0 = [v0;w0];
\end{lstlisting}


\phantomsection
\addcontentsline{toc}{section}{Initiation de Ode}
\subsection*{Initiation de Ode}

\begin{lstlisting}
options=odeset('RelTol',1e-5,'AbsTol',1e-5,'stats','on');
\end{lstlisting}


\phantomsection
\addcontentsline{toc}{section}{Lancement du chronometre}
\subsection*{Lancement du chronometre}

\begin{lstlisting}
tic
\end{lstlisting}


\phantomsection
\addcontentsline{toc}{section}{Appel de Ode (Ode45 ici)}
\subsection*{Appel de Ode (Ode45 ici)}

\begin{lstlisting}
[tout, yout] = ode45(@Impulse,t,u0,options);
\end{lstlisting}
\begin{lstlisting}[language={},xleftmargin=5pt,frame=none]
error: v(0): subscripts must be either integers 1 to (2^63)-1 or logicals
	in:

[tout, yout] = ode45(@Impulse,t,u0,options);

\end{lstlisting}


\phantomsection
\addcontentsline{toc}{section}{Receuil de v (on laisse tomber w)}
\subsection*{Receuil de v (on laisse tomber w)}

\begin{lstlisting}
yout = yout (:,1 :length(z));
\end{lstlisting}
\begin{lstlisting}[language={},xleftmargin=5pt,frame=none]
error: 'yout' undefined near line 1 column 8
	in:

yout = yout (:,1 :length(z));

\end{lstlisting}


\phantomsection
\addcontentsline{toc}{section}{Arret et lecture du chronometre}
\subsection*{Arret et lecture du chronometre}

\begin{lstlisting}
tcpu=toc;
tcpu
\end{lstlisting}
\begin{lstlisting}[language={},xleftmargin=5pt,frame=none]
tcpu =  0.089227

\end{lstlisting}


\phantomsection
\addcontentsline{toc}{section}{Visualisation graphique}
\subsection*{Visualisation graphique}

\begin{lstlisting}
Visualizer(z,t,yout);
\end{lstlisting}
\begin{lstlisting}[language={},xleftmargin=5pt,frame=none]
error: 'yout' undefined near line 1 column 16
	in:

Visualizer(z,t,yout);




\end{lstlisting}


\end{document}
